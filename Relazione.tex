\documentclass[a4paper,10pt]{article}


\usepackage[utf8]{inputenc}
%\usepackage{mathtools}
\usepackage{graphicx}
\usepackage[italian]{babel}
\usepackage{float}
\usepackage{amsmath}
\usepackage{verbatim}
\usepackage{siunitx}

\title{Analisi ponte trifase total-controllato}
\author{Olivieri Daniele}
\date{18 novembre 2019}

\pdfinfo{%
  /Title    (Analisi ponte trifase total-controllato)
  /Author   (Olivieri Daniele)
  /Creator  ()
  /Producer ()
  /Subject  (Elettronica di Potenza)
  /Keywords (FEP, Power Electronics, Three Phase Converter)
}

\begin{document}
\maketitle

\begin{abstract}
 ciao ciao
\end{abstract}

\begin{comment}
\section{Introduzione} %C'è l'abstract come introduzione(?)
Elenco canali oscilloscopio:
1 - Giallo: Vl sul carico
2 - Verde: Vr sulla resistenza o corrente nel carico
3 - Blu: corrente al secondario
4 - Rosso: corrente al primario

Trasformatore stella con neutro al primario - stella al secondario
Riferimenti degli impulsi sulla concatenata
V1/V2 = 9.524
Tensione ingresso 220 V starred
\end{comment}


\section{Norme tecniche che disciplinano la procedura di prova}
Il comitato tecnico che disciplina l'elettronica di potenza è il CT 22 e la norma
di riferimento per la prova è la CEI EN 60146 ``Convertitori a semiconduttore''.

\section{Strumenti utilizzati}
Per effettuare la prova sono stati utilizzati i seguenti strumenti di misura:
\begin{itemize}
 \item Oscilloscopio a 4 canali Keysight DSO-X 2014A
 \item Trasduttore di corrente a 2 canali ad effetto Hall da 5 A, artigianale
\end{itemize}

\section{Componenti utilizzati}
I componenti sottoposti a prova sono sei tiristori \%INSERISCI MODELLO\%.
L'alimentazione del ponte è fornita tramite un trasformatore trifase TTSK0.20 da 200 VA 
conforme alla norma di sicurezza CEI 96-7.

Il trasformatore è collegato a stella con neutro alla rete trifase con una tensione
concatenata di $\SI{220}{\volt_{rms}}$ mentre il secondario è collegato a stella senza
neutro alle tre gambe del ponte.

Il carico è composto da un induttore da $\SI{100}{\milli\henry}$ e
un resistore in serie da $\SI{10}{\ohm}$.



\section{Schema elettrico del circuito di prova}
Il seguente schema rappresenta la struttura trifase in esame collegata ad un
carico puramente resistivo.

I trasduttori di corrente sono stati rappresentati con degli
amperometri. Il voltmetro collegato al canale 1 permette di visualizzare la 
tensione sul carico, il voltmetro collegato al canale 2, tramite una costante di 
attenuazione di valore pari alla resistenza, permette di misurare 
la corrente che circola nel carico.


\begin{figure}[H]
 \centering
 \includegraphics[keepaspectratio=true,width=1\linewidth]{img/circuito_qucs.png}
 % circuito_qucs.png: 1193x395 px, 115dpi, 26.35x8.72 cm, bb=0 0 747 247
 \caption{Struttura e circuito di misura}
 \label{fig:circuito}
\end{figure}

Un unico gate driver trifase appositamente realizzato gestisce il turn-on
dei singoli tiristori.

Il trasformatore trifase è a flusso vincolato dato che sono presenti 
solo tre colonne, questa proprietà non è individuabile dallo schema in cui 
è presente un banco trimonofase a flusso libero.

\section{Richiami teorici}
La rete trifase fornisce una terna di tensioni sinusoidali ad una frequenza di
$\SI{50}{\hertz}$ sfasate tra loro di \ang{120}
così rappresentabili:

\begin{comment}
 \begin{figure}[H]
 \centering
 \includegraphics[keepaspectratio=true,width=1\linewidth]{img/circuito_qucs.png}
 % circuito_qucs.png: 1193x395 px, 115dpi, 26.35x8.72 cm, bb=0 0 747 247
 \caption{Terna trifase}
 \label{fig:circuito}
\end{figure}
\end{comment}

Sia \(\alpha\) l'angolo d'impulso dei componenti, riferito rispetto al valore nullo
delle tensioni concatenate
\begin{equation}
 V_0 = \frac{3}{\pi} \sqrt{2} \sqrt{3} V_\lambda \cos\alpha
 \label{eq:valore_medio_tensione_ponte}
\end{equation}


\section{Descrizione della prova eseguita}
%Inserisci immagini circuito reale

\section{Risultati ottenuti}
%In questa sezione inserisci le immagini dell'oscilloscopio


\end{document}
