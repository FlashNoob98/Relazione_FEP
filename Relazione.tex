\documentclass[a4paper,10pt]{article}


\usepackage[utf8]{inputenc}
%\usepackage{mathtools}
\usepackage{graphicx}
\usepackage[italian]{babel}
\usepackage{float}
%\usepackage{amsmath}
\usepackage{verbatim}

\title{Analisi ponte trifase total-controllato}
\author{Olivieri Daniele}
\date{18 novembre 2019}

\pdfinfo{%
  /Title    (Analisi ponte trifase total-controllato)
  /Author   (Olivieri Daniele)
  /Creator  ()
  /Producer ()
  /Subject  (Elettronica di Potenza)
  /Keywords (FEP, Power Electronics, Three Phase Converter)
}

\begin{document}
\maketitle

\begin{abstract}
 ciao ciao
\end{abstract}

\begin{comment}
\section{Introduzione} %C'è l'abstract come introduzione(?)
Elenco canali oscilloscopio:
1 - Giallo: Vl sul carico
2 - Verde: Vr sulla resistenza o corrente nel carico
3 - Blu: corrente al secondario
4 - Rosso: corrente al primario

Trasformatore stella con neutro al primario - stella al secondario
\end{comment}


\section{Norme tecniche che disciplinano la procedura di prova}
Il comitato tecnico che disciplina l'elettronica di potenza è il CT 22 e la norma
di riferimento per la prova è la CEI EN 60146 ``Convertitori a semiconduttore''.

\section{Strumenti utilizzati}
Per effettuare la prova sono stati utilizzati i seguenti strumenti di misura:
\begin{itemize}
 \item Oscilloscopio a 4 canali Keysight DSO-X 2014A
 \item Trasduttore di corrente ad effetto Hall da 5 A, artigianale
\end{itemize}

\section{Componenti utilizzati}

\section{Schema elettrico del circuito di prova}
Il seguente schema rappresenta la struttura trifase in esame collegata ad un
carico resistivo.

I trasduttori di corrente sono stati schematizzati come semplici
amperometri mentre le sonde collegate ai primi due canali dell'oscilloscopio
coincidono con uno stesso voltmetro perchè collegate in parallelo, le costanti di
amplificazione impostate sull'oscilloscopio sono in realtà diverse perchè i due 
segnali rappresentano grandezze diverse, in questo caso tensione
e corrente sul carico.

\begin{figure}[H]
 \centering
 \includegraphics[keepaspectratio=true,width=1\linewidth]{img/circuito_qucs.png}
 % circuito_qucs.png: 1193x395 px, 115dpi, 26.35x8.72 cm, bb=0 0 747 247
 \caption{Struttura e circuito di misura}
 \label{fig:circuito}
\end{figure}

Ogni singolo tiristore è impulsato autonomamente da un gate driver trifase 
realizzato appositamente.

Il trasformatore trifase è a flusso vincolato dato che sono presenti 
solo tre colonne, questo dettaglio non è individuabile dallo schema.
\section{Richiami teorici}

\section{Descrizione della prova eseguita}

\section{Risultati ottenuti}

\end{document}
